\documentclass[12pt]{amsart}
\usepackage{amsmath}
\usepackage{amssymb}
\usepackage{amsthm}
\usepackage{graphicx}
\usepackage{algorithmic}

\newcommand{\R}{\mathbb{R}}

\newtheorem{thm}{Theorem}[section]
\newtheorem{lem}[thm]{Lemma}

\theoremstyle{remark}
\newtheorem{rem}{Remark}
\newtheorem{defn}{Definition}


\title{Notes on sparse spectral method}
\author{Ryan Compton}
\date{Aug 16, 2010}


\begin{document}
\maketitle

We are interested in approximating $e^{-iHdt}$ via a Chebyshev expansion
$$
e^{-iHdt} = \sum_{k=0}^K a_k \phi_k (-iHdt).
$$
where $\phi_k(\omega) = T_k(-i\omega)$, $\omega \in [-i,i]$ are the complex Chebyshev polynomials. To understand bounds on $K$ we consider the problem of approximating $e^z$ for $z \in [i \lambda_{min} dt, i \lambda_{max} dt]$. Introduce notation
$$
R = \frac{dt}{2}(\lambda_{max} - \lambda_{min})
$$

$$
G = dt \lambda_{min}
$$

$$
\omega = \frac{1}{R}(z - i(R+G)),\omega \in [-i,i]
$$

And write
\begin{eqnarray}
e^{z} & = & e^{i(R+G)}e^{R\omega} \\
& = & e^{i(R+G)}\sum_{k=0}^K C_k J_k(R) \phi_k(\omega)
\end{eqnarray}
where $C_k \in \{1,2 \} $. Note that the Bessel functions of the first kind, $J_k(R)$ vanish for $k>R$ and almost never vanish for $k<R$. We therefore take $K= \alpha R$ with $\alpha >1$ to ensure convergence of the series.


\section{Implications for different timestepping schemes}



%\includegraphics[width=6in]{bessel.eps}



\bibliographystyle{plain}
\bibliography{thebib}

\end{document}
