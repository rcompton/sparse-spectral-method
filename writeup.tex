\documentclass[12pt]{amsart}
\usepackage{amsmath}
\usepackage{amssymb}
\usepackage{amsthm}
\usepackage{graphicx}
\usepackage{algorithmic}

\newcommand{\R}{\mathbb{R}}

\newtheorem{thm}{Theorem}[section]
\newtheorem{lem}[thm]{Lemma}

\theoremstyle{remark}
\newtheorem{rem}{Remark}
\newtheorem{defn}{Definition}


\title{Notes on sparse spectral method}
\author{Ryan Compton}
\date{Aug 16, 2010}


\begin{document}
\maketitle

We are interested in solving the time indpendent Hamiltonian eigenvalue problem
\begin{equation}
H\psi = \lambda \psi
\end{equation}
via a spectral method \cite{FEIT1982}. In classical applications this entails solving the time dependent problem
\begin{equation}
H\psi = \partial_t \psi
\end{equation}
at uniformly spaced points in time with resolution limited by the Nyquist rate
\begin{equation}\label{nyquist}
\Delta t < \pi / V_{max}
\end{equation}
followed by an FFT of $\psi$ into the energy domain to obtain
\begin{equation}
\hat{\psi}(\lambda) = \sum a_n \delta(\lambda - \lambda_n)
\end{equation}
which is sparse. If we can accurately sample the time dependent wavefunction at sparse random locations (ie take some very long timesteps) then we may avoid condition (\ref{nyquist}) by replacing the FFT with $L^1$ minimization. Solving the time dependent problem is accomplished by successively applying complex expoentials
\begin{equation}
\psi(t) = e^{-iH\Delta t}\psi(0).
\end{equation}
It turns out that this can (probably) be done to spectral accuracy by expanding $e^{-iH\Delta t}$ into a Chebyshev polynomial series and applying the kinetic energy operator by multiplication in $k$-space \cite{Fritz1983} \cite{Fritz1983}.

\section{Notes}

I'm stuck on the propagators that are not Strang splitting. I'm guessing that the problem is how I discretize k-space

Also, be careful about off by 1 errors in Pt!!!

\bibliographystyle{plain}
\bibliography{thebib}

\end{document}
